\documentclass[a4paper,12pt]{scrartcl}
\usepackage[sexy]{evan}
\usepackage{fontspec}
\usepackage{tikz,pgfplots}
\title{\textbf{CALCULUS AND ANALTYIC  GEOMETRY 1(MTH1101)}}
\setmainfont[Scale = 1.4]{TH SarabunPSK}
\declaretheorem[style=thmbluebox,name={Identity}]{ID}
\renewcommand{\theID}{\Roman{ID}}
\author{Sarawut Suebsang}
%forimage
\pgfplotsset{compat=1.15}
\usepackage{mathrsfs}
\usetikzlibrary{arrows}
\pagestyle{empty}

\begin{document}

	\maketitle
\section{\protect \text{Overviews}}
limit, continuty, derivative, chain rule, implicit differrentiation, higher-order derivative, differential, antiderivative, definite integral, area between curves, derivative and integral of transcendental function, indeterminate form, L's Hopital's rule, extreme value, concavity, curve sketching, realated rate

\section{Limit} 
\begin{example}
	พิจารณา $f: \mathbb{R} \rightarrow \mathbb{R}$  นิยามโดย $Ex1$
\end{example}
\begin{figure}[h!]
	\centering
	\definecolor{ttttff}{rgb}{0.2,0.2,1}
\begin{tikzpicture}[line cap=round,line join=round,>=triangle 45,x=1cm,y=1cm]
	\begin{axis}[
		x=1cm,y=1cm,
		axis lines=middle,
		ymajorgrids=true,
		xmajorgrids=true,
		xlabel = X,
		ylabel = Y,
		xmin=-4,
		xmax=4,
		ymin=-1,
		ymax=5,
		xtick={-4,-3,...,4},
		ytick={1,2,5},]
		\clip(-4,-1) rectangle (5,5);
		\draw [samples=50,rotate around={0:(0,1)},xshift=0cm,yshift=1cm,line width=2pt,color=ttttff,domain=-4:4)] plot (\x,{(\x)^2/2/0.5});
		\node[below right] at(2,5) {$(2,5)$};
	\end{axis}

\end{tikzpicture}
	\caption{Ex1}
	\label{fig:f1}
\end{figure}
\begin{proof}
	$f(2) = 2^2+1 = 5 $ จากรูปจะสังเกตที่ $f$ ใกล้ๆ $2$ ค่าของฟังก์ชันจะใกล้ๆ $5$ ด้วย จะกล่าวได้ว่า
	$\lim_{x \to 2} f(x) = 5$
\end{proof}



\begin{example}
	พิจารณา $g(x) = \frac{x^2-1}{x-1}$
\end{example}

\begin{figure}[h!]
	\centering
	\definecolor{ududff}{rgb}{0.30196078431372547,0.30196078431372547,1}
	\definecolor{ffffff}{rgb}{1,1,1}
	\definecolor{qqqqff}{rgb}{0,0,1}
	\begin{tikzpicture}[line cap=round,line join=round,>=triangle 45,x=1cm,y=1cm]
		\begin{axis}[
			x=1cm,y=1cm,
			axis lines=middle,
			ymajorgrids=true,
			xmajorgrids=true,
			xlabel=X,
			ylabel=Y,
			xmin=-4,
			xmax=4,
			ymin=-1,
			ymax=4,
			xtick={-4,-3,...,4},
			ytick={-1,0,...,4},]
			\clip(-13.064008801856069,-10.231358489840877) rectangle (6.481640114065647,6.413522742445793);
			\draw [line width=2pt,color=qqqqff,domain=-13.064008801856069:6.481640114065647] plot(\x,{(--1--1*\x)/1});
			\begin{scriptsize}
				\draw[color=qqqqff] (-10.836633616207743,-9.94646166376958) node {$f$};
				\draw [fill=ffffff] (1,2) circle (2.5pt);
				\draw[color=ffffff] (1.1462995530940852,2.364534517371909) node {$A$};

			\end{scriptsize}
		\end{axis}
	\end{tikzpicture}
	\caption{Ex2}
\label{fig:f2}
\end{figure}
\begin{proof}
	$ g(x)= \frac{(x-1)(x+1)}{(x-1)} = x+1; x \not = 1$ จะได้กราฟเส้นตรงที่มีจุดโป๋ที่จุด $x=1$  $g(1)$ จะไม่มีค่า เเต่ $\lim_{x \to 1} g(x) =2$
\end{proof}

\begin{example}
	พิจารณา $h(x) = \begin{cases}\frac{x^2-1}{x-1} , & ; x \not=1 \\
		1 ,& ; x=1
	\end{cases} $
\end{example}

\begin{figure}[h!]
	\centering
	\definecolor{ududff}{rgb}{0.30196078431372547,0.30196078431372547,1}
	\definecolor{ffffff}{rgb}{1,1,1}
	\definecolor{qqqqff}{rgb}{0,0,1}
	\begin{tikzpicture}[line cap=round,line join=round,>=triangle 45,x=1cm,y=1cm]
		\begin{axis}[
			x=1cm,y=1cm,
			axis lines=middle,
			ymajorgrids=true,
			xmajorgrids=true,
			xlabel=X,
			ylabel=Y,
			xmin=-4,
			xmax=4,
			ymin=-1,
			ymax=4,
			xtick={-4,-3,...,4},
			ytick={-1,0,...,4},]
			\clip(-13.064008801856069,-10.231358489840877) rectangle (6.481640114065647,6.413522742445793);
			\draw [line width=2pt,color=qqqqff,domain=-13.064008801856069:6.481640114065647] plot(\x,{(--1--1*\x)/1});
			\begin{scriptsize}
				\draw[color=qqqqff] (-10.836633616207743,-9.94646166376958) node {$f$};
				\draw [fill=ffffff] (1,2) circle (2.5pt);
				\draw[color=ffffff] (1.1462995530940852,2.364534517371909) node {$A$};
				\draw [fill=ududff] (1,1) circle (2.5pt);
				\draw[] (1.1462995530940852,1.3630790075455328) node {$(1,1)$};
			\end{scriptsize}
		\end{axis}
	\end{tikzpicture}
	\caption{Ex3}
\label{fig:f3}
\end{figure}

\begin{proof}
	จะได้ $h(x) = \begin{cases}x+1 , & ; x \not=1 \\
		1 ,& ; x=1
	\end{cases} $ \\
$h(1) = 1$ เเต่ $\lim_{x \to 1} h(x) = 2$
\end{proof}

\begin{example}
	พิจารณา $a(x) = \begin{cases}
		0 & ; x<0 \\
		1 &; x \ge 0
	\end{cases}$
\end{example}
\begin{figure}[h!]
	\centering
	\definecolor{ffffff}{rgb}{1,1,1}
	\definecolor{ududff}{rgb}{0.30196078431372547,0.30196078431372547,1}
	\definecolor{xdxdff}{rgb}{0.49019607843137253,0.49019607843137253,1}
	\begin{tikzpicture}[line cap=round,line join=round,>=triangle 45,x=1cm,y=1cm]
		\begin{axis}[
			x=1cm,y=1cm,
			axis lines=middle,
			ymajorgrids=true,
			xmajorgrids=true,
			xlabel=X,
			ylabel=Y,
			xmin=-4,
			xmax=4,
			ymin=-1,
			ymax=2,
			xtick={-4,-3,...,4},
			ytick={-1,-0,...,2},]
			\clip(-13.064008801856067,-10.231358489840877) rectangle (6.481640114065649,6.413522742445793);
			\draw [line width=2pt,color=xdxdff] (0,1)-- (6.447107165450946,0.974583335630128);
			\draw [line width=2pt,color=xdxdff] (0,0)-- (-12.131619189259094,0);
			\begin{scriptsize}
				\draw [fill=xdxdff] (0,1) circle (2.5pt);
				\draw[color=xdxdff] (0.1448440432677075,1.3630790075455328) node {$A$};
				\draw [fill=ududff] (6.447107165450946,0.974583335630128) circle (2.5pt);
				\draw[color=ududff] (6.308975370992136,1.3458125332381814) node {$B$};
				\draw[color=black] (3.287342367205651,0.9141506755543984) node {$f$};
				\draw [fill=ffffff] (0,0) circle (2pt);
				\draw[color=ffffff] (0.1448440432677075,0.34435702341180496) node {$C$};
				\draw [fill=xdxdff] (-12.131619189259094,0) circle (2.5pt);
				\draw[color=xdxdff] (-11.993487394800283,0.37888997202650754) node {$D$};
				\draw[color=black] (-6.002020810149369,0.4824888178706155) node {$g$};
			\end{scriptsize}
		\end{axis}
	\end{tikzpicture}
\end{figure}
\begin{proof}
	$a(0)=1$ เเต่ $\lim_{x \to 0} a(x)$ ไม่มีค่า เนื่องจากดูทางซ้ายเเละขวาเเล้วมีค่าไม่เท่ากัน
\end{proof}

\begin{theorem*}[Limit Theorem]
	ให้ $k$ เป็นค่าคงที่, $I$ เป็นช่วง, $f: I \rightarrow \mathbb{R}, g: I \rightarrow \mathbb{R}, c$ เป็นจุดลิมิตของ $I$ เเละ \[ \lim_{x \to c} f(x) = L \; \text{เเละ} \; \lim_{x \to c} g(x) = M \]
	\begin{itemize}
		\item $\lim_{x \to c} k = k$ 
		\item $ \lim_{x \to c} x = c$ 
		\item $ \lim_{x \to c} [f(x)+g(x)]  = L+M$ 
		\item $ \lim_{x \to c} [f(x)-g(x)]  = L-M$ 
		\item $ \lim_{x \to c} f(x)g(x) = LM$
		\item $\lim_{x \to c} \frac{f(x)}{g(x)} =\frac{L}{M} ; M \not = 0$
		\item $\lim_{x \to c} f(x)^{\frac{m}{n}} = L^{\frac{m}{n}} ; m,n \in \mathbb{N} $ เเละ $L^{\frac{m}{n}} \in \mathbb{R}$
	\end{itemize}
\end{theorem*}
\begin{example}
	$\lim_{x \to 2} (5x-3)$
\end{example}
\begin{proof}
	$= (\lim_{x \to 2} 5x) - ( \lim_{ x \to 2} 3) = 5(\lim_{x \to 2} x) - (\lim_{x \to 2} 3) =5(2)-3=7$
\end{proof}
\begin{example}
	$\lim(2-3x)$
\end{example}
\begin{proof}
	$= (\lim_{x \to 1 } 2)  - (\lim_{x \to -1} 3x) = (\lim_{x \to -1} 2) - 3(\lim_{x \to -1}x) =2-3(-1)=5$	
\end{proof}
\begin{example}
	$\lim_{x \to -2} x^2$
\end{example}
\begin{proof}
	$=\lim_{x \to -2} xx = (\lim_{x \to -2} x)(\lim_{x \to -2} x) = (-2)(-2)=4$
\end{proof}
\begin{example}
	$\lim_{x \to 2}(x^2+3x+1)$
\end{example}
\begin{proof}
	$= \lim_{x \to 2} x^2 + \lim_{x \to 2} 3x + \lim_{x \to 2} 1 = (\lim_{x \to 2} x)^2 +3(\lim_{x \to 2} x) +1 = 2^2 +3(2)+1=11$
\end{proof}

\begin{example}
	$\lim_{x \to 2} \frac{x^2+1}{2x}$
\end{example}

\begin{proof}
	$= \frac{\lim_{x \to 2} (x^2+1)}{\lim_{x \to 2} 2x} =  \frac{\lim_{x \to 2} x^2+\lim_{x \to 2} 1}{2\lim_{x \to 2} x} =  \frac{2^2+1}{2(2)}=\frac{5}{4}$
\end{proof}
\begin{example}
	$\lim_{x \to 3} \frac{x^2-3}{x-3}$
\end{example}
\begin{proof}
	พิจารณา $\lim_{x \to 3} (x^2-3) = 3^2-3 =6$ เเละ $\lim_{x \to 3} (x-3) = 3-3 =0$ ดังนั้น $\lim_{x \to 3} \frac{x^3-3}{x-3} $ ไม่มีค่า
\end{proof}

\begin{example}
	$\lim_{x \to 4} \frac{\sqrt{x^2+9}}{2x}$
\end{example}
\begin{proof}
	$\frac{\lim_{x \to 4}\sqrt{x^2+9}}{\lim_{x \to 4}2x} = \frac{ \sqrt{\lim_{x \to 4}(x^2+9)}}{\lim_{x \to 4}2x} = \frac{5}{8}$
\end{proof}
\begin{example}
	$\lim_{x \to 2} \frac{x^2-4}{x-2}$
\end{example}
\begin{proof}
	$ =\lim_{x \to 2} \frac{(x-2)(x+2)}{(x-2)} = \lim_{x \to 2} (x+2) = 4$
\end{proof}
\begin{example}
	$\lim_{x \to 2} \frac{x^3-8}{x-2}$
\end{example}
\begin{proof}
	$ =  \lim_{x \to 2} \frac{(x-2)(x^2+2x+4)}{x-2} = \lim_{x \to 2}(x^2+2x+4) =12$  $(A^3-B^3=(A-B)(A^3+AB+B^3),A^3+B^3=(A+B)(A^2-AB+B^2)))$
\end{proof}
\begin{example}
	$\lim_{x \to 0} \frac{3x-5x^2}{x}$
\end{example}
\begin{proof}
	$= \lim_{x \to 0} \frac{x(2-5x)}{x}= \lim_{x \to 0} (2-5x) =2 $
\end{proof}
\begin{example}
	$\lim_{x\to 3} \frac{x^2+x-6}{x-2}$ 
\end{example}
\begin{proof}
	$= \lim_{x\to 3} \frac{(x+3)(x-2)}{x-2} = \lim_{x\to 3}(x+3) =5$
\end{proof}
\begin{example}
	$\lim_{x \to 1} \frac{3-2x-x^2}{2x^2-x-1}$
\end{example}
\begin{proof}
$ = \lim_{x\to 1} \frac{(3+x)(1-x)}{(2x+1)(x-1)}=\lim_{x \to 1} -\frac{3+x}{2x+1} = -\frac{4}{3}$
\end{proof}
\begin{example}
	$\lim_{x \to 0} \frac{\sqrt{x+4}-2}{x}$
\end{example}
\begin{proof}
$=\lim_{x \to 0} [\frac{\sqrt{x+4}-2}{x} \times \frac{\sqrt{x+4}+2}{\sqrt{x+4}+2} ] =\lim_{x \to 0} \frac{\sqrt{x+4}^2-2^2}{x(\sqrt{x+4}+2)}=\lim_{x \to 0} \frac{x+4-4}{x(\sqrt{x+4}+2)}= \lim_{x \to 0} \frac{x}{x(\sqrt{x+4}+2)} =\lim_{x \to 0} \frac{1}{\sqrt{x+4}+2} = \frac{1}{4} $
\end{proof}
\begin{example}
	$\lim_{x\to 2} \frac{\sqrt{2x-3}-1}{x-2}$
\end{example}
\begin{proof}
	พิจารณา$ \frac{\sqrt{2x-3}-1}{x-2} = \frac{\sqrt{2x-3}-1}{x-2} \times \frac{\sqrt{2x-3}+1}{\sqrt{2x-3}+1} = \frac{(2x-3)-1}{(x-2)(\sqrt{2x-3}+1)} = \frac{2(x-2)}{(x-2)(\sqrt{2x-3}+1)}=\frac{2}{(\sqrt{2x-3}+1)} \\ ; x \not = 2$ ดังนั้น $\lim_{x\to 2} \frac{\sqrt{2x-3}-1}{x-2} =\lim_{x\to 2} \frac{2}{(\sqrt{2x-3}+1)} = 1 $
\end{proof}
\begin{example}
	$\lim_{x \to 4} \frac{4x-x^2}{2-\sqrt{x}}$
\end{example}
\begin{proof}
	พิจารณา $\frac{4x-x^2}{2-\sqrt{x}}=\frac{x(4-x)}{x-\sqrt{x}}=\frac{x(2^2-\sqrt{x}^2)}{2-\sqrt{x}}=\frac{x(2+\sqrt{x})(2-\sqrt{x})}{2-sqrt{x}} = x(2+\sqrt{x})$ \\
	ดังนั้น $\lim_{x \to 4} \frac{4x-x^2}{2-\sqrt{x}} = \lim_{x \to 4} x(2+\sqrt{x}) =  16$
\end{proof}
\subsection{one-sided limit}
\[f(x) = \begin{cases}
	3 & ; x \ge 1 \\
	2 & ; x < 1
\end{cases} \]
\begin{figure}[h!]
	\centering
	\definecolor{ffffff}{rgb}{1,1,1}
	\definecolor{qqqqff}{rgb}{0,0,1}
	\definecolor{ududff}{rgb}{0.30196078431372547,0.30196078431372547,1}
	\begin{tikzpicture}[line cap=round,line join=round,>=triangle 45,x=1cm,y=1cm]
		\begin{axis}[
			x=1cm,y=1cm,
			axis lines=middle,
			ymajorgrids=true,
			xmajorgrids=true,
			xmin=-4,
			xmax=4,
			ymin=-1,
			ymax=4,
			xlabel=X,
			ylabel=Y,
			xtick={-4,-3,...,4},
			ytick={-1,-3,...,4},]
			\clip(-15.16,-12.08) rectangle (7.48,7.2);
			\draw [line width=2pt,color=qqqqff] (1,3)-- (9,3);
			\draw [line width=2pt,color=qqqqff] (1,2)-- (-10,2);
			\begin{scriptsize}
				\draw [fill=ududff] (1,3) circle (2.5pt);
				\draw [fill=ududff] (9,3) circle (2.5pt);
				\draw [fill=ffffff] (1,2) circle (2.5pt);
				\draw [fill=ududff] (-10,2) circle (2.5pt);
			\end{scriptsize}
		\end{axis}
	\end{tikzpicture}
\end{figure}
ให้ $c \in \mathbb{R} $
\begin{itemize}
	\item $c>1 : \lim_{x \to c} f(x) \lim_{x \to c} 3 = 3 $ \item  $ c<1 : \lim_{x \to c} f(x) \lim_{x \to c} 2 = 2 $
	\item $c=1 : \lim_{x \to 1} f(x) $ ไม่มีค่า
\end{itemize}
Limit เเบบต่างๆ\\
ลิมิตขวา right-handed limit : $\lim_{x \to 1^+} f(x) = \lim_{x \to 1^+} 3 =3$ \\
ลิมิตซ้าย left-handed limit : $\lim_{x \to 1^-} f(x) = \lim_{x \to 1^-} 2 =2$ \\
\underline{ข้อสังเกต} :  ถ้า limit ด้านเดียว มีค่าเท่ากันทั้งสองด้าน นั่นคือ $ \lim_{x\to c^-} f(x) = \lim_{x \to c^+} f(x) = L$ จะได้ว่า $\lim_{x \to c} f(x) =L$
\begin{example}
	หา $\lim_{x\to -2} f(x)$
\end{example}
\begin{figure}[h!]
	\definecolor{qqqqff}{rgb}{0,0,1}
	\definecolor{ududff}{rgb}{0.30196078431372547,0.30196078431372547,1}
	\definecolor{ffffff}{rgb}{1,1,1}
	\centering
	\begin{tikzpicture}[line cap=round,line join=round,>=triangle 45,x=1cm,y=1cm]
		\begin{axis}[
			x=1cm,y=1cm,
			axis lines=middle,
			ymajorgrids=true,
			xmajorgrids=true,
			xmin=-4,
			xmax=4,
			ymin=-4,
			ymax=5,
			xlabel=X,
			ylabel=Y,
			xtick={-4,-3,...,4},
			ytick={-4,-3,...,5},]
			\clip(-10.298513395051963,-9.222094779570922) rectangle (10.5650993578638,6.91957672429007);
			\draw [->,line width=2pt,color=qqqqff] (-2,-3) -- (4.537089252687497,6.718643054117527);
			\draw [->,line width=2pt,color=qqqqff] (-2,4) -- (-5.760761343655356,-2.8759396966214243);
			\begin{scriptsize}
				\draw [fill=ffffff] (-2,-3) circle (2.5pt);
				\draw [fill=ududff] (4.537089252687497,6.718643054117527) circle (2.5pt);
				\draw [fill=ududff] (-2,4) circle (2.5pt);
				\draw [fill=ududff] (-5.760761343655356,-2.875939696621424) circle (2.5pt);
			\end{scriptsize}
		\end{axis}
	\end{tikzpicture}
\end{figure}
\begin{proof}
	$f(-2) = 4, \lim_{x \to -2^-} f(x)=4, \lim_{x \to -2^+} f(x) = -3$
\end{proof}
\begin{example}
	\[f(x) = \begin{cases}
		x^2-1 &; x<0 \\2x &; x\ge 0	\end{cases}\]
\end{example}
\begin{proof}
	$f(0)  = 2(0) = 0, \lim_{x \to 0^-} f(x) = \lim_{x \to 0^-} (x^2-1) = 0^2-1=-1, \lim_{x \to 0^+} f(x) = \lim_{x \to 0^+} 2x = 2(0) = 0$ ดังนั้น $\lim_{x\to 0}$ ไม่มีค่า เพราะ left กับ right-handed limit มีค่าไม่เท่ากัน
\end{proof}

\begin{example}
	$f(x) = \frac{|x|}{x} = \begin{cases}
		1 &; x>0 \\-1 &; x<0	\end{cases}$
\end{example}
\begin{proof}
	$\lim_{x\to 0^+} f(x) = \lim_{x \to 0^+} 1 = 1, \lim_{x \to 0^-} f(x) = \lim_{x \to 0^-} -1 =-1 $ ดังนั้น $\lim_{x \to 0} f(x)$ ไม่มีค่า , $f(0)$ ไม่มีค่า (domain $f$ คือ $\mathbb{R}-\{0\} $)
\end{proof}
\subsection{infinite limit}
\begin{example}
	$f(x)= \frac{1}{x}$
\end{example}

\begin{figure}[h!]
	\centering
	\definecolor{zzttqq}{rgb}{0,0,1}
	\begin{tikzpicture}[line cap=round,line join=round,>=triangle 45,x=1cm,y=1cm]
		\begin{axis}[
			x=1cm,y=1cm,
			axis lines=middle,
			ymajorgrids=true,
			xmajorgrids=true,
			xmin=-4,
			xmax=4,
			ymin=-4,
			ymax=4,
			xlabel=X,
			ylabel=Y,
			xtick={-4,-3,...,4},
			ytick={-4,-3,...,4},]
			\clip(-10.298513395051963,-8.954183219340864) rectangle (10.5650993578638,6.91957672429007);
			\draw[line width=2pt,color=zzttqq,samples=1000,domain=-10.298513395051963:10.5650993578638] plot(\x,{1/(\x)});
			\begin{scriptsize}
				\draw[color=zzttqq] (-10.147813142422555,-0.15496291303489787) node {$f$};
			\end{scriptsize}
		\end{axis}
	\end{tikzpicture}
\end{figure}

\begin{proof}
	$Dom(f) = (-\infty,0) \cup (0,\infty) = \mathbb{R}\textbackslash \{0\}$ \\
	$\lim_{x \to 0} f(x) $ ไม่มีค่า $\lim_{x \to 0^+} f(x) = +\infty$ (ไม่มีค่า ไม่ใช่จำนวน), $\lim_{x \to 0^-} f(x) = -\infty $  (ไม่มีค่า)
\end{proof}
\begin{example}
	$\lim_{x \to c^+} \frac{1}{(x-c)^p}$
\end{example}
\begin{proof}
	ถ้าให้ $t= x-c$ จะได้ว่า $t \to c^+$ ดังนั้น $\lim_{x \to c^+} \frac{1}{(x-c)^p}=\lim_{t \to 0^+}\frac{1}{t^p}=\infty$
\end{proof}
\begin{example}
	$\lim_{x \to -3^+} \frac{1}{(x+3)^2}$
\end{example}
\begin{proof}
	$=\lim_{t \to 0^-} \frac{1}{t^2} = \lim_{s \to 0^+} \frac{1}{s} = \infty$
\end{proof}

\begin{example}
	$\lim_{x \to -3^-} \frac{1}{(x+3)^5}$
\end{example}
\begin{proof}
	$= \lim_{t \to 0^-} \frac{1}{t^5} = \lim_{s \to 0^-} \frac{1}{s} = -\infty$
\end{proof}
\begin{example}
	$\lim_{x \to 0^+}\frac{1}{x^{\frac{2}{5}}}$
\end{example}
\begin{proof}
	$=\lim_{x \to 0^+} \frac{1}{(x^2)^{\frac{1}{5}}}=\lim_{x \to 0^+} (\frac{1}{x^2})^{\frac{1}{5}} = (\lim_{x \to 0^+} \frac{1}{x^2})^{\frac{1}{5}}=\infty$
\end{proof}
\begin{example}
	$\lim_{x \to 2^-} \frac{1}{(x-2)^3}$
\end{example}
\begin{proof}
	$=\lim_{t\to 0^-} \frac{1}{t^3}=-\infty$
\end{proof}
\subsubsection{vertical asymtote}
\begin{example}
	f(x) = $\frac{1}{x} ; x>0$
\end{example}
\begin{figure}[h!]
	\centering
	\definecolor{ffqqqq}{rgb}{1,0,0}
\definecolor{qqqqff}{rgb}{0,0,1}
	\begin{tikzpicture}[line cap=round,line join=round,>=triangle 45,x=1cm,y=1cm]
		\begin{axis}[
			x=1cm,y=1cm,
			axis lines=middle,
			ymajorgrids=true,
			xmajorgrids=true,
			xmin=-1,
			xmax=4,
			ymin=-1,
			ymax=4,
			xlabel=X,
			ylabel=Y,
			xtick={-1,-3,...,4},
			ytick={-1,-3,...,4},]
			\clip(0,-4.081541717656686) rectangle (12.926069982391187,11.792218225974253);
			\draw[line width=2pt,color=qqqqff,samples=100,domain=-7.937542770524577:12.926069982391187] plot(\x,{1/(\x)});
			\draw [line width=2pt,color=ffqqqq] (0,-4.081541717656686) -- (0,11.792218225974253);

		\end{axis}
	\end{tikzpicture}
\end{figure}
\begin{proof}
	เราจะได้เเกน $Y$ (เส้นตรง $X = 0$) เป็นเส้นกำกับเเนวดิ่งของ $f$
\end{proof}
\begin{definition*}
	จะเรียกเส้นตรง $x=c$ ว่าเส้นตรงกำกับเเนวดิ่ง (vertical asymtote) ของ $f$ ก็ต่อเมื่อข้อใดข้อหนึ่งต่อไปนี้เป็นจริง
	\begin{itemize}
		\item $\lim_{x \to c^+ f(x)} = \infty$
		\item $\lim_{x \to c^-} f(x) = \infty$
		\item $\lim_{x \to c^+} f(x) = -\infty$
		\item $\lim_{x \to c^-} f(x) = -\infty$
	\end{itemize}
\end{definition*}
\begin{example}
	พิจารณา $f(x)= \frac{1}{x-1}$
\end{example}
\begin{figure}[h!]
	\centering
	\definecolor{ffqqqq}{rgb}{1,0,0}
	\definecolor{qqqqff}{rgb}{0,0,1}
	\begin{tikzpicture}[line cap=round,line join=round,>=triangle 45,x=1cm,y=1cm]
		\begin{axis}[
			x=1cm,y=1cm,
			axis lines=middle,
			ymajorgrids=true,
			xmajorgrids=true,
			xmin=-4,
			xmax=4,
			ymin=-4,
			ymax=4,
			xlabel=X,
			ylabel=Y,
			xtick={-4,-3,...,4},
			ytick={-4,-3,...,4},]
			\clip(-10.097219662558352,-6.814933818776231) rectangle (4.134242291187941,4.0128558538975145);
			\draw[line width=2pt,color=qqqqff,samples=1000,domain=-10.097219662558352:4.134242291187941] plot(\x,{1/((\x)-1)});

		\end{axis}
	\end{tikzpicture}
\end{figure}
\begin{proof}
	ถ้า $c \not = 1$ จะได้ว่า $\lim_{x \to c} f(x) = \frac{1}{c-1}$ เเต่ $\lim_{c\to 1^+} f(x) = \lim_{x \to 1^+} \frac{1}{x-1} = \infty$
	จึงได้ว่า เส้นตรง $x=1$ เป็นเส้นกำกับเเนวดิ่งของ $y = \frac{1}{x-1}$ เพียงเส้นเดียว
\end{proof}

\begin{example}
	$f(x)= \frac{1}{x^2}$
\end{example}
\begin{figure}[h!]
	\centering
	\definecolor{ffqqqq}{rgb}{1,0,0}
	\definecolor{qqqqff}{rgb}{0,0,1}
	\begin{tikzpicture}[line cap=round,line join=round,>=triangle 45,x=1cm,y=1cm]
		\begin{axis}[
			x=1cm,y=1cm,
			axis lines=middle,
			ymajorgrids=true,
			xmajorgrids=true,
			xmin=-4,
			xmax=4,
			ymin=-1,
			ymax=4,
			xlabel=X,
			ylabel=Y,
			xtick={-4,-3,...,4},
			ytick={-1,-0,...,4},]
			\clip(-10.097219662558352,-6.575077718432193) rectangle (4.134242291187941,4.0128558538975145);
			\draw[line width=2pt,color=qqqqff,samples=100,domain=-10.097219662558352:4.134242291187941] plot(\x,{1/(\x)^(2)});

			\draw [line width=2pt,color=ffqqqq] (0,-6.575077718432193) -- (0,4.0128558538975145);
		\end{axis}
	\end{tikzpicture}
\end{figure}
\begin{proof}
	ถ้า $c \not = 0$ เเล้ว $\lim_{x \to c} \frac{1}{x^2}=\frac{1}{c^2} \in \mathbb{R}$ พิจารณา $\lim_{x \to 0^+} f(x) = \lim_{x \to 0^+}\frac{1}{x^2} = \infty$ ดังนั้น $f$ มีเส้นกำกับเเนวดิ่งเพียงเส้นเดียว คือ เส้นตรง $x=0$
\end{proof}
\begin{example}
	$\lim_{x \to 1^+} \frac{x+5}{x-1}$
\end{example}
\begin{proof}
	พิจารณา $\lim_{x \to 1^+} (x+5) =6, \lim_{x \to 1^+} (x-1)=0$ จะได้ $\lim_{x \to 1^+}\frac{1}{x-1}=\infty $ ดังนั้น $\lim_{x \to 1^+} \frac{x+5}{x-1}= \infty$
\end{proof}
\begin{example}
	$\lim_{x \to 1^-} \frac{x+5}{x-1}$
\end{example}
\begin{proof}
	พิจารณา $\lim_{x \to 1^-} (x+5) =6, \lim_{x \to 1^-} (x-1)=0$ จะได้ $\lim_{x \to 1^-}\frac{1}{x-1}=-\infty $ ดังนั้น $\lim_{x \to 1^-} \frac{x+5}{x-1}= -\infty$
\end{proof}
\begin{example}
	$\lim_{x \to 4^+}\frac{1-x}{x-4}$
\end{example}
\begin{proof}
	เนื่องจาก $\lim_{x \to 4^+} (1-x) = -3$ เเละ $\lim_{x \to 4^+} \frac{1}{x-4} = \infty$ ดังนั้น $\lim_{x \to 4^+} \frac{1-x}{x-4} = - \infty$
\end{proof}

\begin{example}
	$\lim_{x \to 4^-}\frac{1-x}{x-4}$
\end{example}
\begin{proof}
	เนื่องจาก $\lim_{x \to 4^-} (1-x) = -3$ เเละ $\lim_{x \to 4^-} \frac{1}{x-4} =- \infty$ ดังนั้น $\lim_{x \to 4^+} \frac{1-x}{x-4} =  \infty$
\end{proof}
\begin{example}
	พิจารณา $\lim_{x \to 1^-} \frac{x+3}{(x-1)^2}$
\end{example}
\begin{proof}
	ลองคำนวณ $\lim_{x \to 1^-} (x+3) =4$ เเละคำนวณ $\lim_{x \to 1^-} \frac{1}{(x-1)^2} = \infty$ ดังนั้น $\lim_{x \to 1^- } \frac{x+3}{(x-1)^2} = \infty$
\end{proof}

\begin{example}
	คำนวน $\lim_{x \to 1^+} \frac{1}{x(x-1)}$
\end{example}
\begin{proof}
	พิจารณา $\lim_{x \to 1^+} \frac{1}{x} =1, \lim_{x \to 1^+} \frac{1}{x-1}=\infty$ ดังนั้น  $\lim_{x \to 1^+} \frac{1}{x(x-1)} = \infty$
\end{proof}
\subsection{Limit at infinity}

\end{document}
