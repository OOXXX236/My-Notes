\documentclass[a4paper,12pt]{article}
\usepackage{fontspec}
\usepackage{siunitx}
\setmainfont[Scale=1.4]{TH SarabunPSK}
\author{S. Suebsang}
\title{\textbf{Algebra Problems}}
\usepackage{fancyhdr}
\pagestyle{fancy}
\usepackage{amsmath,amsthm,amssymb}
\usepackage{graphicx}
\newtheorem{problem}{Problem}
\begin{document}
	
	
	\maketitle
	
	
	%a+b+c=0
	
	
	\begin{problem}
		จงเเสดงว่า ถ้า $a+b+c=0$ เเล้ว $a^3+b^3+c^3=3abc$
	\end{problem}
	\begin{problem}
		จงเเยกตัวประกอบ $(x-y)^3+(y-z)^3+(z-x)^3$
	\end{problem}
	\begin{problem}
		จงเเยกตัวประกอบ $(2x-y-z)^3+(2y-z-x)^3+(2z-x-y)^3$
	\end{problem}
	\begin{problem}
		ให้ $r$  เป็นคำตอบของสมการ $\sqrt[3]{r}+\frac{1}{\sqrt[3]{r}}=1$ จงหาค่าของ $r^3+\frac{1}{r^3}$
	\end{problem}
	\begin{problem}
		จงหาจำนวนจริงทั้งหมดที่สอดคล้องกับสมการ $\sqrt[3]{x+2}-\sqrt[3]{x-2}=1$
	\end{problem}
	\begin{problem}
		จงพิสูจน์ว่า $\sqrt[3]{2+\sqrt{5}}+\sqrt[3]{2-\sqrt{5}}$ เป็นจำนวนเต็ม
	\end{problem}
	\begin{problem}
		จงหาจำนวนจริง $x,y,z$ ทั้งหมดที่สอดคล้องระบบสมการ
		$$x+y+z=6$$
		$$x^2+y^2+z^2=14$$
		$$x^3+y^3+z^3=36$$
	\end{problem}
	\begin{problem}
		ให้ $a,b,c$ เป็นจำนวนจริง โดยที่ $\sqrt[3]{a-b}+\sqrt[3]{b-c}+\sqrt[3]{c-a}=0$ จงเเสดงว่า ${a,b,c}$ มีสมาชิกน้อยกว่า $3$
	\end{problem}
	\begin{problem}
		จงเเยกตัวประกอบ $(x+2y-3z)^3+(y+2z-3x)^3+(z+2z-3y)^3$
	\end{problem}
	\begin{problem}
		ให้ $x=7+5\sqrt{2}$ จงหาค่าของ $\sqrt[3]{x}-\frac{1}{\sqrt[3]{x}}$
	\end{problem}
	\begin{problem}
		จงหาคำตอบของระบบสมการ 
		$$x+y+z=0$$
		$$x^3+y^3+z^3=12$$
		$$x^6+y^6+z^6=264$$
	\end{problem}
	\begin{problem}
		ถ้า $a$ เเละ $b$ เป็นจำนวนจริงซึ่ง $$\sqrt[3]{a}-\sqrt[3]{b}=12$$ เเละ $$ab=(\frac{a+b+8}{6})^3$$ จงหาค่าของ $a-b$
	\end{problem}
	\begin{problem}
		ให้ $a,b$ เป็นรากของสมการ 
		$$\frac{(x-2563)^3+(x-2020)^3}{(2x-4583)^3}=(\frac{2563}{2020})^3\times(\frac{2^6\times{5^3}\times{101^3}}{11^3\times{233^3}})$$
	\end{problem}
	\begin{problem}
		จงเเก้สมการ
		$$(2015x-2014)^3=(2x-2)^3+(2013x-2012)^3$$
	\end{problem}

	%a^2+b^2=0
	
	
	
	\begin{problem}
		ให้ $x$ เป็นจำนวนเต็มบวก จงหาค่าต่ำสุดของ $x+\frac{1}{x}$
	\end{problem}
	\begin{problem}
		ให้ $a,b,c,d$ เป็นจำนวนจริงบวก จงหาค่าต่ำสุดของ $\frac{a}{b+d}+\frac{b}{c+a}+\frac{c}{d+b}+\frac{d}{a+c}$
	\end{problem}
	\begin{problem}
		ให้ $a,b,c$ เป็นจำนวนจริงซึ่ง $a^2+b^2+c^2=ab+bc+ca$ จงหาค่าของ $$(\frac{a+b-c}{c})^2+(\frac{a-b+c}{b})^2+(\frac{-a+b+c}{a})^2$$
	\end{problem}
	\begin{problem}
		จงหาจำนวนจริงทั้งหมดที่สอดคล้องสมการ $1-2^x-3^x+4^x-6^x+9^x=0$
	\end{problem}
	\begin{problem}
		ให้ $a$ เเละ $b$ เป็นจำนวนจริง โดยที่ $(\sqrt{a^2+1}-a)(\sqrt{b^2+1}+b)=1$ จงหาค่าของ $a-b$
	\end{problem}
	\begin{problem}
		จงหาค่า $k$ ที่น้อยที่สุดที่ทำให้สมการต่อไปนี้เป็นจริง
		$$a^2+2b^2+3c^2+4d^2=a+2b+3c+4d+5k$$
	\end{problem}
	\begin{problem}
		จงหาคำตอบที่เป็นจำนวนจริงของระบบสมการต่อไปนี้
		$$x(1+y+y^2)=1+x+x^2$$
		$$y(1+z+z^2)=1+y+y^2$$
		$$z(1+x+x^2)=1+z+z^2$$
	\end{problem}
	\begin{problem}
		ให้ $a,b,c,d$ เป็นจำนวนจริงซึ่ง $$1+a^2+b^2+c^2+d^2=a+b+c+d$$จงหาค่าของ
		$$a^2+b^2+c^2$$
	\end{problem}
	\begin{problem}
		จงหาค่าต่ำสุดของ $$x^3+x^2+x+\frac{1}{x}+\frac{1}{x^2}+\frac{1}{x^3}$$ เมื่อ $x>0$
	\end{problem}
	\begin{problem}
		ให้ $a,b,c$ เป็นจำนวนจริง ซึ่ง $a^2+b^2+c^2= \frac{2}{3}(ab+bc+ca)$ จงหาค่าของ $a+b+c$
	\end{problem}
	\begin{problem}
		จงหาคำตอบที่เป็นจำนวนจริงของระบบมการต่อไปนี้
		$$x(1+y+y^2)=1+z+z^2$$
		$$y(1+z+z^2)=1+x+x^2$$
		$$z(1+x+x^2)=1+y+y^2$$
	\end{problem}
	\begin{problem}
	ให้ $a,b,c$ เป็นจำนวนจริงบวกซึ่งสอดคล้องกับระบบสมการ
	$$\frac{8a^2}{a^2+9}=b,\frac{10b^2}{b^2+16}=c,\frac{6c^2}{c^2+25}=a$$
	
	\end{problem}

	\begin{problem}
		จงเเก้ระบบสมการ 
		$$a(b-c+1)=b^2-bc+c$$
		$$b(c-a+1)=c^2-ca+a$$
		$$c(a-b+1)=a^2-ab+b$$
	\end{problem}



	\begin{problem}
		ให้ $x$ เป็นจำนวนจริงจงหาค่าต่ำสุดที่เป็นไปได้ของ 
		$$(x-1)^2+(x-2)^2+..+(x-5)^2$$
	\end{problem}
	\begin{problem}
		จงหาค่าต่ำสุดของ $\frac{5-4x+x^2}{2-x}$ เมื่อ $x<2$
	\end{problem}
	\begin{problem}
		กำหนด $a,b,c,d$ เป็นจำนวนจริง ซึ่ง
		$$a^2+b^2+c^2+d^2 = 3a+8b+24c+37d , 2018=3b+8c+24d+37a$$ 
	\end{problem}

	%factor tactics
	
	
	\begin{problem}
		จงเเยกตัวประกอบพหุนาม $x^4-1$
	\end{problem}

	\begin{problem}
		จงหาจำนวนจริง $x$ ซึ่งสอดคล้องกับสมการ $$4^x+4^{-x}=2^x+2^{-x}$$
	\end{problem}
	\begin{problem}
		จงหาคำตอบของสมการ $$(x+1)(x+2)(x+3)(x+4)=1$$
	\end{problem}
	\begin{problem}
		จงหาจำนวนจริง $k$ ที่ทำให้ สมการพหุนาม $$(x+1)(x+2)(x+3)(x+4)=k$$ มีรากเป็นจำนวนจริงทั้งหมด
	\end{problem}
	\begin{problem}
		ถ้า $(x+2)(x+4)(x+6)(x+8)=945$ เเละ $x$ เป็นจำนวนจริงเเล้ว $x$ เท่ากับเท่าใด
	\end{problem}
	\begin{problem}
		จงหารากที่เป็นจำนวนจริงของ 
		$$(x+1)(x-2)(x+6)(x-3)=28x^2$$
	\end{problem}
	\begin{problem}
		จงหาคำตอบของระบบสมการ $$\frac{2^x+2^{-x}}{3^x+3^{-x}}=\frac{4^x+4^{-x}}{9^x+9^{-x}}$$
	\end{problem}
	\begin{problem}
		จงหาคำตอบที่เป็นจำนวนจริงของระบบสมการ 
		$$2z=w+\frac{13}{w},2w=x+\frac{13}{x},2x=y+\frac{13}{y}, 2y=z+\frac{13}{z}$$
	\end{problem}

	\begin{problem}
		จงหารากทั้งหมดที่สอดคล้องกับสมการ $x^4+16x-12=0$
	\end{problem}


	\begin{problem}
		กำหนดให้ $x,y,z$ เป็นจำนวนบวกซึ่ง
		$$ x+y+xy=8$$
		$$ y+z+yz=15$$
		$$z+x+zx=35 $$
		จงหาค่าของ $x+y+z+xy$
	\end{problem}


	%change variable tatics
	
	
	

	\begin{problem}
		จงหาจำนวนจริง $x$ ทั้งหมดที่ทำให้ $x^3+1=2\sqrt[3]{2x-1}$
	\end{problem}
		\begin{problem}
		กำหนด $x,y,z$ เป็นจำนวนจริงจงเเก้ระบบสมการ
		$$(x-1)(y-1)(z-1)=xyz-1$$
		$$(x-2)(y-2)(z-2)=xyz-2$$
	\end{problem}
	\begin{problem}
		ให้ $x,y$ เป็นจำนวนจริงซึ่ง
		$$3^x+(\frac{1}{4})^y=31$$
		$$9^x+(\frac{1}{2})^y=731$$
		จงหา $(x,y)$
	\end{problem}

	%definition
	

	%ยังเเยกเเนวไม่ได้
	\begin{problem}
		สัมประสิทธิ์ของ $x^{88}$ จากการคูณ
		$$(x+1)(x+2)(x-3)(x+4)(x+5)(x-6)...(x+88)(x+89)(x-90)$$ มีค่าเท่าใด
	\end{problem}
	\begin{problem}
		กำหนด $S=1+\frac{1}{\sqrt{2}}+\frac{1}{\sqrt{3}}+...+\frac{1}{\sqrt{1000000}}$ ส่วนที่เป็นจำนวนเต็มของ $\frac{S}{2}$ มีค่าเท่าใด
	\end{problem}


	\begin{problem}
		มีสามสิ่งอันดับ $(x,y,z)$ ที่เป็นจำนวนจริงซึ่ง 
		$$ x+y^2=z^3,x^2+y^3=z^4, x^3+y^4=z^5 $$ ทั้งหมดมีกี่อันดับ
	\end{problem}
	\begin{problem}
		ให้ $x$ เป็นจำนวนจริง ซึ่ง $$\frac{\sqrt{x+54}+\sqrt{x}}{\sqrt{x+54}-\sqrt{x}} = \frac{\sqrt{x-1}+\sqrt{2}}{\sqrt{x-1}-\sqrt{2}}$$ จงหา $x$
	\end{problem}
	\begin{problem}
		$$\frac{2563\sqrt{x+4}+2020\sqrt{x}}{2563\sqrt{x+4}-2020\sqrt{x}}=\frac{2563\sqrt{x-3}+2020\sqrt{5}}{2563\sqrt{x-3}-2020\sqrt{5}}$$
	\end{problem}
	\begin{problem}
		ให้ $x$ เป็นจำนวนจริง ซึ่ง 
		$$ \frac{1+x+\sqrt{x^2-x+1}}{2-x+\sqrt{x^2-x+1}}=\frac{x}{1-x}$$
	\end{problem}
	\begin{problem}
		กำหนด $a,b,c$ ซึ่ง
		 $$ a^2+b^2+c^2 = 61 $$
		 $$ab+bc+ca  =54$$
		 $$a^2b^2+b^2c^2+c^2a^2 = 1044$$
		 จงหา $a^3+b^3+c^3$
	\end{problem}
	\begin{problem}
		กำหนด $x,y$ ซึ่ง
		$$x^3-3xy^2=23$$
		$$y^3-3x^2y=10\sqrt{2}$$
		จงหา $x^2+y^2$
	\end{problem}
	\begin{problem}
		จงหาสัมประสิทธิ์ของ $x^7$ จากการกระจายพหุนาม 
		$$(6x^3+5x^2-4x+1)^3$$
	\end{problem}
	\begin{problem}
		ให้ $x,y$ เป็นจำนวนจริงจงเเก้ระบบสมการ
		$$x+y+\sqrt{x^2-y^2}=12$$
		$$y\sqrt{x^2-y^2}=12$$
	\end{problem}
	\begin{problem}
		สำหรับจำนวนจริง $x,y$ ใดๆ กำหนดฟังก์ชัน $f$ ดังนี้
		$$f(x)=(2x)^3-30(x-1)(2x-3)$$
		จงหาค่าของ $f(2563)+f(-2558)$
	\end{problem}
	\begin{problem}
		กำหนด $a,b,c$ ซึ่ง
		$$a^2+(b-30)(c-26)=309$$
		$$b^2+(c-28)(a-31)=262$$
		$$c^2+(a-26)(b-27)=582$$
		จงหา $a+2b+3c$
	\end{problem}
	\begin{problem}
		กำหนด $x,y,z$ ซึ่ง
		$$x+y+z=2$$
		$$x^2+y^2+z^2=3$$
		$$xyz=4$$
		จงหาค่าของ $\frac{1}{xy+z-1}+\frac{1}{yz+x-1}+\frac{1}{zx+y-1}$
	\end{problem}
	\begin{problem}
		จงหาค่าของ $$(1-\frac{1}{1+2})(1-\frac{1}{1+2+3})...(1-\frac{1}{1+2+...+2563})$$
	\end{problem}
	\begin{problem}
		ให้ $a,b$ เป็นจำนวนจริงจงเเก้ระบบสมการ
		$$a+\frac{3a-b}{a^2+b^2}=3$$
		$$b-\frac{a+3b}{a^2+b^2}=0$$ 
	\end{problem}
	\begin{problem}
		ถ้า $a^{4x} = \sqrt{97-7\sqrt{192}}$ จงหาค่าของ
		$\frac{a^{6x}+a^{-6x}}{a^{2x}+a^{-2x}}$
	\end{problem}
	\begin{problem}
		ให้ $x,y$ เป็นจำนวนจริง ซึ่ง $$x^3+y^3=1957$$เเละ
		$$(x+y)(x+1)(y+1)=2014$$ จงหาค่าของ $x+y$
	\end{problem}
	\begin{problem}
		ให้ $x,y$ เป็นจำนวนจริงซึ่ง $$(x-\sqrt{x^2-2018})(y-\sqrt{y^2-2018})=2018$$ จงหาค่าของ $5x^2-4y^2+3x-3y-2017$
	\end{problem}
	\begin{problem}
		ให้ $x,y$ เป็นจำนวนจริง จงเเก้ระบบสมการ 
		$$x(x+3)(x+y)=8$$
		$$x^2+4x=6-y$$
	\end{problem}
	\begin{problem}
		หารากที่เป็นจำนวนจริงของสมการ $$(x^2-x+1)=(x^2+x+1)(x^2+2x+4)$$
	\end{problem}
	\begin{problem}
		กำหนด $x,y,z$ ซึ่ง 
		$$x-\sqrt{yz}=-15 $$
		$$y-\sqrt{zx}=3 $$
		$$z-\sqrt{xy}=21$$
		จงหา $61(x+y+z)$
	\end{problem}
	\begin{problem}
		ให้ $x,y$ เป็นจำนวนจริง ซึ่ง 
		$$x+y=4$$
		$$|z+1|=xy+2y-9$$
		จงหา $x+2y+3z$
	\end{problem}
	\begin{problem}
		ให้ $k$ เป็นจำนวนจริงซึ่ง 
		$$\sqrt{k+\sqrt{k}}- \sqrt{k-\sqrt{k}}=\frac{4}{3}\sqrt{\frac{k}{k+\sqrt{k}}}$$ 
		จงหาผลรวมของ $k$
	\end{problem}
	\begin{problem}
		จงหาจำนวนจริง $x$ ซึ่ง $\sqrt{3-\sqrt{3+x}}=x$
	\end{problem}
	\begin{problem}
		กำหนดให้ 
		$$x=\frac{2}{\sqrt{400}-\sqrt{396}}$$
		$$y=\frac{3}{\sqrt{900}-\sqrt{891}}$$
		หา $(11x^{2020})(233y^{2020})$
	\end{problem}
	\begin{problem}
		กำหนด $a,b,c$ ซึ่ง
		$$\frac{x}{3}+\frac{y}{5}-\frac{z}{7}=12 $$
		$$\frac{x}{18}-\frac{y}{10}+\frac{z}{8}=5$$
		$$\frac{x}{2}-\frac{y}{20}-\frac{z}{28}=14$$
		จงหา $y(z-x)$
	\end{problem}
	\begin{problem}
		หา $n$ เป็นจำนวนเต็มบวก ซึ่ง
		$$\frac{(3(n+1))!(n+19)!}{(n+21)!(3n+1)!}=\frac{n}{9}$$
	\end{problem}
	\begin{problem}
		$\frac{2020^4+3\times2020^3-3\times2020^2+6\times2020+8}{2020^3-2020^2+2020+2}$
	\end{problem}
	\begin{problem}
		กำหนด $x,y$ ซึ่ง
		$$(x+\frac{1}{y})(y+\frac{1}{x})=7$$
		หา 
		$$(x^2+\frac{1}{y^2})(y^2+\frac{1}{x^2})$$
	\end{problem}
	\begin{problem}
		หาจำนวนจริง $a,b,c$ ซึ่ง 
		$$\sqrt{a-4b+52}+2\sqrt{b-4c+64}+3\sqrt{c-4a+42}+12\sqrt{3}\sqrt{a+b+c}=158$$
	\end{problem}
	\begin{problem}
		หา $x$ ที่เป็นจำนวนจริง ซึ่ง
		$$\frac{x-30}{60}+\frac{x-60}{30}= \frac{60}{x-30}+\frac{30}{x-60}$$
	\end{problem}
	\begin{problem}
		หา $a$ เป็นจำนวนจริง ซึ่ง 
		$$\frac{a^2+8a+20}{a+4}-\frac{a^2+6a+12}{a+3}=\frac{a^2+4a+6}{a+2}-\frac{a^2+2a+2}{a+1}$$
	\end{problem}
	\begin{problem}
		กำหนด $a,b,c$ ซึ่ง 
		$$a=\sqrt{2}+\sqrt{3}-\sqrt{5}$$
		$$b=\sqrt{2}-\sqrt{3}+-\sqrt{5}$$
		$$c=-\sqrt{2}+\sqrt{3}+\sqrt{5}$$
		หา
		$$\frac{a^4}{(a-b)(a-c)}+\frac{b^4}{(b-c)(b-a)}+\frac{c^4}{(c-a)(c-b)}$$
	\end{problem}
	\begin{problem}
		หารากทั้งหมดของสมการ 
		$$\frac{13x-x^2}{x+1}(x+\frac{13-x}{x+1})=42$$
	\end{problem}
	\begin{problem}
		ใกำหนด $a,b,c$ ซึ่ง
		$$\frac{ab}{a+b}=\frac{1}{3}$$
		$$\frac{bc}{b+c}=\frac{1}{4}$$
		$$\frac{ca}{c+a}=\frac{1}{5}$$
		หา
		$$\frac{24abc}{ab+bc+ca}$$
	\end{problem}
	\begin{problem}
		ให้ $y$ เป็นจำนวนจริง ซึ่ง
		$$(y+7)(y-3)+(\frac{12}{y+2})^2=0$$
		หาผลรวมกำลังสองของ$y $ ที่เป็นไปได้
	\end{problem}
	\begin{problem}
		ให้จงหา $x,y,z$ เป็นจำนวนจริงบวก ซึ่ง
		$$x(8-y)=16$$
		$$y(8-z)=16$$
		$$z(8-x)=16$$
	\end{problem}

\begin{problem}
		ให้ หา $x,y,z$ เป็นจำนวนจริง ซึ่ง
	$$ x^3+\frac{1}{3}y = x^2+x-\frac{4}{3}$$
	$$y^3+\frac{1}{4}z=y^2+y-\frac{5}{4}$$
	$$z^3+\frac{1}{5}x=z^2+z-\frac{6}{5}$$
\end{problem}

\begin{problem}
	ให้ $p,q$ เป็นจำนวนจริง ซึ่ง
	$$2563p-5126\sqrt{p}+10252q-10252\sqrt{q}+10252\sqrt{pq}=7689$$
	หา $$\frac{\sqrt{p}+2\sqrt{q}+2560}{4-\sqrt{p}-2\sqrt{q}}$$
\end{problem}


\end{document}