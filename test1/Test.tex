\documentclass[a4paper,12pt]{scrartcl}
\usepackage[sexy]{evan}
\usepackage{fontspec}
\usepackage{asymptote}
\declaretheorem[style=thmbluebox,name={Identity}]{ID}
\renewcommand{\theID}{\Roman{ID}}
\title{\textbf{HOMEWORK}}
\setmainfont[Scale = 1.4]{TH SarabunPSK}
\author{Sarawut Suebsang}
\begin{document}
	
	\maketitle
	
	\begin{ID}
		
		\[ a^3 + b^3 + c^3 - 3abc = (a + b + c)(a^2 + b^2 + c^2 - ab - bc - ca)  \]
		\label{A}
		
	\end{ID}

จากเอกลักษณ์ \ref{A} จะเห็นได้ว่า $a^3 + b^3 + c^3 - 3abc = 0 \iff a + b + c = 0\; \textrm{หรือ} \; a = b = c$ 

\begin{example}
	
	ให้ $p$ เป็นจำนวนเฉพาะ จงหาคู่อันดับ $(x , y)$ ของจำนวนนับทั้งหมด ซึ่ง \[ x^3 + y^3 - 3xy = p - 1 \]
	
\end{example}

\begin{proof}
	
	กำหนดให้ $z = 1$ จะเขียนสมการใหม่ได้เป็น $x^3 + y^3 + z^3 - 3xyz = p$ นั่นคือ
		\[ (x + y + z)(x^2 + y^2 + z^2 - xy - xz - yz) = p\]
		เนื่องจาก $x , y , z$ ทำให้ $x + y + z \ge 3$ ดังนั้น $x + y + z = p$ เเละ $x^2 + y^2 + z^2 - xy - xz - yz = 1$
		\begin{align*}
			x^2 + y^2 + z^2 - xy - xz - yz &= 1 \\
			(x - y)^2 + (x - z)^2 + (y - z)^2 &= 2 
		\end{align*}
เนื่องจาก $x , y \ge 1$ กรณี $x ,y \ge 2$ ทั้งคู่จะเป็นได้กรณีเดียวคือ $x = y =2$ กรณี $x$ หรือ $y$ เท่า $1$ โดยไม่เสียนัยให้ $y = 1$ จะได้ $x = 2$ เเต่ $x + y + z = 4$ ไม่เป็นจำนวนเฉพาะดังนั้นกรณีนี้ไม่เกิดจะได้คำตอบ $(x , y) = (2 , 2)$ เท่านั้น   

\end{proof}

\begin{ID}
	กำหนดให้ $abc = 1$ จะได้
	\[  \frac{1}{ab+a+1} + \frac{1}{bc+b+1}  + \frac{1}{ca+c+1} = 1 \]
	\label{B}
\end{ID}

จากเงื่อนไข $abc = 1$ เราจะสามารถเเทน $ \textstyle a = \frac{x}{y}, b = \frac{y}{z}, c = \frac{z}{x}$

\begin{example}
	ให้ $a , b , c$ เป็นจำนวนจริงบวก ซึ่ง $abc = 1$ จงเเสดงว่า \[ \cycsum \frac{2}{(a + 1)^2 + b^2 + 1} \le 1 \]
\end{example}

\begin{proof}
	
	จาก $(a + 1)^2 + b^2 + 1 = (a^2+b^2) + 2a + 2$ โดย AM-GM เเละใช้ \ref{B} จะได้
	
	\begin{align*}
		\cycsum \frac{2}{(a + 1)^2 + b^2 + 1} &\le  \cycsum \frac{1}{ab+a+1}\\
		\cycsum \frac{2}{(a + 1)^2 + b^2 + 1} &\le 1
	\end{align*}
จะเท่ากันเมื่อ $a = b = c$ จาก AM-GM

\end{proof}

\begin{theorem}[Euler]
	ให้ $a , n$ เป็นจำนวนเต็ม เเละ $\gcd{( a, n)} =1$ จะได้ \[ a^{\phi{(n)}} \equiv 1 \pmod{n} \]
\end{theorem}

$\phi{(n)} =\#  \{ a \in \NN : 1 \le a \le n \textrm{เเละ} \gcd{(a , b)} = 1 \} $

\begin{example}
	
	พิจารณาจำนวนนับ $n \ge 2,$ เศษจากการหาร $2^{2^n}$ ด้วย $2^n - 1$  จะอยู่ในรูปกำลังของ $4$\\ จงเเสดงว่าข้อความข้างต้นจริงถ้าไม่จริงยกตัวอย่างค้าน 
	
\end{example}

\begin{proof}
	ไม่จริง เช่น $n = 25$ จะได้ 
	\begin{align*}
		2^{25} &\equiv 1 \pmod{2^{25} - 1}\\	2^{2^{25}} &\equiv 2^{2^{25}  \pmod{25}} \pmod{2^{25} - 1}
	\end{align*}

จากออยเลอร์ จะได้ $2^{20} \equiv 1 \pmod{25} $ ดังนั้น 
\[ 2^{2^{25}} \equiv 2^{7} \pmod{2^{25} - 1} \] ซึ่ง $2^7$ ไม่เป็นกำลังของ $4$
	

\end{proof}

\begin{ID}[Binomial coefficient
	]
	\[ (x + y)^n = \sum_{k=0}^{n} { n \choose k}x^{n-k}y^k \]
	\label{C}
\end{ID}

\begin{ID}
	\[ {n+1 \choose k+1} = {n \choose k} + {n \choose k+1}  \]
	\label{D}
\end{ID}

\begin{ID}
	\[ {n+1 \choose k+1} = \frac{n+1}{r+1}{n \choose r} \]\label{E}
\end{ID}

\begin{example}
	
	จงเเสดงว่า \[ \sum_{j = 1}^{n} \frac{(-1)^{j+1}}{j} {n\choose j} = 1 + \frac{1}{2} + \frac{1}{3} + \dots +\frac{1}{n} \]

\end{example}

\begin{proof}
	เนื่องจาก ${n \choose j} = 0 $ ถ้า $j > n$ ดังนั้นจากโจทย์เขียนใหม่ได้เป็น \[ S_n = \sum_{j = 1}^{\infty} \frac{(-1)^{j+1}}{j} {n\choose j} = 1 + \frac{1}{2} + \frac{1}{3} + \dots +\frac{1}{n} \]
	เราจะอุปนัยบน $n$ ให้ $P(n)$ เเทนสมการที่โจทย์ให้เเสดง \\
	\underline{\bfseries{ขั้นฐาน}} เห็นได้ชัดว่า $P(1)$ จริง \\
	\underline{\bfseries{ขั้นอุปนัย}} สมมติ $P(n)$ จริง เมื่อ $k \ge 1$ จะเเสดง $P(n+1)$ จริง \\
	$P(n+1)$ จริง $\iff S_{n+1} - S_{n} = \frac{1}{n+1}$
	\begin{align*}
		&\iff \sum_{j = 1}^{\infty} \frac{(-1)^{j+1}}{j} \left[{n+1\choose j} - {n\choose j }\right ]=  \frac{1}{n+1}  \\
		&\iff \sum_{j = 1}^{\infty} \frac{(-1)^{j+1}}{j} {n\choose j-1} =  \frac{1}{n+1}  (\textrm{จาก}
		\ref{D})\\
		&\iff \sum_{j = 1}^{\infty}(-1)^{j+1}{n+1 \choose j} = 1 \; \;  (\textrm{จาก}
		\: \ref{E} \: \textrm{เเละเป็นจริงจากเเทน} \: \ref{C} \:  \textrm{ด้วย} \: x = 1, y=-1)
	\end{align*}
	ดังนั้น $P(n+1)$ เป็นจริงจากหลักอุปนัยเชิงคณิตศาสตร์จึงสรุปได้ว่า $P(n)$ เป็นจริงทุก $n \in \NN$
\end{proof}

\begin{lemma}
	ให้ $ n \in \NN, n+1| n! \iff n+1$ เป็นจำนวนประกอบ
	\label{F}
\end{lemma}
\begin{proof}
	ให้ $n \in \NN$ \\
 $(\Rightarrow)$ \; สมมุติ \; $n+1 | n!$ \\
 จะพิสูจน์โดยหาข้อขัดเเย้ง สมมุติ \; $n+1$ \; ป็นจำนวนเฉพาะ \\ จะได้ \; $\gcd{(n + 1, i)} = 1 $ เมื่อ $i = 1 ,2 ,3,\dots,n$ \\
 นั่นคือ \; $\gcd{(n+1,n!)} = 1$ \; ซึ่งขัดเเย้ง  
 ดังนั้น $n+1$ เป็นจำนวนประกอบ \\
 $(\Leftarrow)$ สมมุติ $n$ เป็นจำนวนประกอบ \\
 ดังนั้นจะมี $a , b$ ซึ่ง $n+1=ab$ เเละ $1< a \le b < n+1$ \\
 กรณี $a \neq b
  $ จะได้ $n+1 | 1 \times \dots \times a  \dots \times b \times \dots \times n = n!$ \\
 กรณี $a = b$ จะได้ $n+1 | 1 \times \dots \times a  \dots \times 2a \times \dots \times n = n!$ \\
 ดังนั้น $n+1| n!$
 
\end{proof} 

\begin{example}
	
	หาคู่อันดับของจำนวนนับ $(n , k)$ ทั้งหมด ซึ่ง \[ (n + 1)^k = n! + 1
	\]
	
\end{example}
\begin{proof}
	จากโจทย์ $n+1 \not| n! $ เเล้วใช้ \ref{F} จะได้ $n+1$ เป็นให้จำนวนเฉพาะ ให้ $p = n+1$ \\
	เขียนใหม่ได้เป็น $p^k = (p-1)! +1$ ซึ่งจะรูปเเล้วจะได้ \[ p^{k-1} + p^{k-2} +...+p+1 = (p-2)! \]
	\underline{กรณี $p-1$ เป็นจำนวนเฉพาะหรือหนึ่ง }จะสรุปได้ว่า $(n,k)=(2,1),(1,1)$ ซึ่งสอดคล้องกับโจทย์ \\
	\underline{กรณี $p - 1$ เป็นจำนวนประกอบ }จาก \ref{F} ได้ $p-1 | (p-2)!$ ทำให้ \begin{align*} p^{k-1} + p^{k-2} +...+p+1 &\equiv 0 \pmod{p-1} \\
		k &\equiv 0 \pmod{p-1} \\
		k &= pq \; \textrm{บาง} \; q \in \NN
	 \end{align*}
 เเทน $k$ กลับไปสมการเเรกจะได้ $ p^{pq} = (p-1)! +1$ จาก $(p-1)! \le (p-1)^p$ \\
 ทำให้ $p^{pq}<(p-1)^p+1$ จะเห็นได้ว่าถ้า $q>1$ จะขัดเเย้งดังนั้น $q=1$ \\
 พิจารณา $ p^p-(p-1)^p = p^{p-1}+p^{p-2}{p-1} +...+(p-1)^{p-1}  >1$ ขัดเเย้ง \\
 ดังนั้นคำตอบ $(n,k) = (1,1),(2,1)$ เท่านั้น
\end{proof}

\begin{example}
	
	ให้ $x , y , z \in \RR^+$ จงเเสดงว่า \[ \sqrt{x(y + 1)} + \sqrt{y(z + 1)} + \sqrt{z(x + 1)} \le \frac{3}{2}\sqrt{(x + 1)(y + 1)(z + 1)} \]
	
\end{example}

\end{document}


