\documentclass[a4paper,12pt]{article}
\usepackage{fontspec}
\usepackage{siunitx}
\setmainfont[Scale=1.4]{TH SarabunPSK}
\author{S. Suebsang}
\title{\textbf{เเนวข้อสอบคัดเข้าค่าย 3 วลัยลักษณ์}}
\usepackage{fancyhdr}
\pagestyle{fancy}
\usepackage{amsmath,amsthm,amssymb}
\usepackage{graphicx}
\newtheorem{problem}{Problem}
\begin{document}
	\maketitle
	\begin{problem}
		ให้ $a,b \in Z$ ซึ่ง $a^2+b^2$ หารด้วย $a+b$ ได้ผลหารคือ $r$ จงหา $a,b$ ทั้งหมดที่เป็นไปได้ซึ่ง $q^2+r=2017$
	\end{problem}
	\begin{problem}
		ให้ $P(x)=(x+d_1)(x+d_2)...(x+d_9)$ เมื่อ $d_1,d_2,..,d_9$ เป็นจำนวนเต็มที่ต่างกัน จงพิสูจน์ว่า "มีจำนวนเต็ม $N$ ซึ่งสำหรับทุกจำนวนเต็ม $x\ge N, p|P(x)$ บางจำนวนเฉพาะ $p\ge 20$ "
	\end{problem}
	\begin{problem}
		ให้ $a,b \in N$ จงพิสูจน์ว่า ถ้า $4ab-1|(4a^2-1)^2$ เเล้ว $a=b$
	\end{problem}
	\begin{problem}
		ให้ $ABC$ เป็นสามเหลี่ยมซึ่ง $AB = AC$ เเละให้ $D$ เป็นจุดกึ่งกลางของ $AC$ เส้นเเบ่งครึ่งมุม $\angle{BAC}$ ตัดวงกลมที่ล้อมรอบ $D,B$ เเละ $C$ ที่จุด $E$ ซึ่งอยู่ภายใน $\bigtriangleup{ABC}$  เส้นตรง $BD$ ตัดวงกลมที่ล้อมรอบ $A,E$ เเละ $B$ ที่จุด $B$ เเละ $F$ เส้นตรง $AF$ เเละ $BE$ ตัดกันที่จุด $I$ เเละ เส้นตรง $CI$ เเละ $BD$ ตัดกันที่จุด $K$ จงเเสดงว่า $I$ เป็นจุดกึ่งกลางของวงกลมเเนบใน $\bigtriangleup{KAB}$
	\end{problem}
	\begin{problem}
		$a,b,c \in R^+$ จงเเสดงว่า $$\frac{a}{c}+\frac{b}{a} +\frac{c}{b} \le \frac{a^9+b^9+c^9}{a^3b^3c^3}$$
	\end{problem}
	\begin{problem}
		$a,b,c \in R^+,a+b+c =1$ จงเเสดงว่า $$\sqrt{a+bc}+\sqrt{b+ca}+\sqrt{c+ab} \le 2$$
	\end{problem}
	\begin{problem}
		พิจารณาจุดทุกจุดในระนาบ ระบายสีจุดเหล่านั้นด้วยปากกาซึ่งมีอยู่ $2$ สีได้เเก่ ปากกาสีเเดง เเละ ปากกาสีน้ำเงิน จงพิสูจน์ว่ามีรูปสามเหลี่ยมที่มีความยาวด้าน $1,2\sqrt{3}$ หน่วย ที่มีจุดยอดทั้งสามเป็นสีเดียวกันทั้งหมด
	\end{problem}
	\begin{problem}
		กำหนดให้ $S = {1,2,..,16}$ ทำการเเบ่งเซต $S$ ออกเป็น $3$ เซตย่อย $S=A\cup{B}\cup{C}$ โดยที่ $A\cap{B}=B\cap{C}=C\cap{A}=\emptyset$ เเละ $A,B,C$ ไม่ใช่เซตว่าง 
		จงพิสูจน์ว่า จะต้องมีอยู่ $1$ เซตย่อยเสมอจาก $A,B,C$ ที่บรรจุเลข $x,y,z$(อาจซ้ำกันได้) ซึ่ง $x+y=z$ 
	\end{problem}
	\begin{problem}
		กำหนดสี่เหลี่ยมที่มีวงกลมล้อมรอบได้ $ABCD$ ให้ด้าน $|AB|=|CD|=a , |AD|=|BC|=b , |AC|=m$ เเละ $|BD|=n$ ถ้า $a^4+b^4=m^2n^2$ จงหา $\angle{A}$
	\end{problem}
	\begin{problem}
		$f:N\rightarrow{N} ; f(n)+f(n+1)=f(n+2)f(n+3)-(p-1)$ เมื่อ $p$ เป็นจำนวนเฉพาะ จงหาฟังก์ชัน $f$
	\end{problem}
	\begin{problem}
		$f:Z \rightarrow Z ; f(m+n)+f(mn-1)=f(m)f(n)$ จงหาฟังก์ชัน $f$
	\end{problem}
	\begin{problem}
		$f:R \rightarrow R ; f(xy)=xf(x)+yf(y)$ จงหาฟังก์ชัน $f$
	\end{problem}
	\begin{problem}
		ให้ $p,q,r \in R^+$ เเละ $p+q+r=1$ จงเเสดงว่า $$7(pq+qr+rp) \le 2+9pqr$$
	\end{problem}
	\begin{problem}
		ให้ $a,b,c \in R^+$ เเละ $a+b+c=1$ จงเเสดงว่า $$\frac{a+b}{ab}+\frac{b+c}{bc}+\frac{c+a}{ca} \ge 18$$
	\end{problem}
	\begin{problem}
		ให้ $p,q,r,x,y,z \in R^+$ จงเเสดงว่า $$3(x+y+z)(\frac{a^3}{x}+\frac{b^3}{y}+\frac{c^3}{z}) \ge (a+b+c)^3$$
	\end{problem}
	\begin{problem}
		ให้ $a,b,c,d$ เป็นจำนวนจริงบวกใดๆ ซึ่ง $a+c=b+d$ จงเเสดงว่า $$\sum_{cyc}\sqrt{(1+ab)^2+b^2c^2} \ge 2\sqrt{3}(a+c)$$
	\end{problem}
	\begin{problem}
		กำหนดให้ $t\in R^+, 0<a<b$ จงเเสดงว่า 
		\begin{itemize}
			\item ถ้า $a+t>0$ เเล้ว $(1+\frac{t}{b})^b \ge (1+\frac{t}{a})^a$
			\item ถ้า $a-t>0$ เเล้ว $(1-\frac{t}{a})^{-a} \ge (1-\frac{t}{b})^{-b}$
		\end{itemize}
	\end{problem}
	\begin{problem}
		หา $a,b,c$ เป็นจำนวนเต็มบวก ซึ่ง
		\begin{itemize}
			\item $3^a+4^b = 5^c$
			\item $5^a+12^b=13^c$
		\end{itemize}
	\end{problem}
	\begin{problem}
	สำหรับเเต่ละจำนวนเต็มบวก $n \ge 2$ จงเเสดงว่า $x-1$ เป็นตัวประกอบของพหุนาม $P_0(x), P_1(x), \dots , P_{n-2}(x)$ ถ้าพหุนาม $x^{n-1} +x^{n-2} +\dots+x+1$ หารพหุนาม $P_0(x^n) + xP_1(x^n)+\dots+x^{n-2}P_{n-2}(x^n)$
	\end{problem}

\end{document}
