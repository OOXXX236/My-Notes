\documentclass[a4paper,12pt]{article}
\usepackage{fontspec}
\usepackage{siunitx}
\setmainfont[Scale=1.4]{TH SarabunPSK}
\author{S. Suebsang}
\title{\textbf{Problems from TON NAM}}
\usepackage{fancyhdr}
\pagestyle{fancy}
\usepackage{amsmath,amsthm,amssymb}
\usepackage{graphicx}
\newtheorem{problem}{Problem}
\begin{document}
	\maketitle
	\begin{problem}
		กำหนดให้ $ABCD$ เป็นสี่เหลี่ยมจตุรัสมีพื้นที่ 40, $E$, $F$ เป็นจุดบน ส่วนของเส้นตรง $AB,BC$ ตามลำดับโดยที่ $BE = \frac{AB}{3}$ เเละ $BF = \frac{2BC}{5}, EC$ ตัด $BD,FD$ที่ $G,H$ ตามลำดับ จงหาพื้นที่ของสี่เหลี่ยม $BGHF$
	\end{problem}
	\begin{problem}
		(Zenith GM8) สามเหลี่ยมมุมเเหลม $ABC$ มีวงกลมเเนบใน มีจุดศูนย์กลางที่ $I$ เเละสัมผัส $\overline{BC},\overline{CA},\overline{AB}$ ที่จุด $D,E,F$ ตามลำดับ ให้ เส้นตรง $DI$ ตัด $\overline{EF}$ ที่จุด $K$ ถ้า $BC = 34, DI = 10, IK = 4$ จงหาพื้นที่ ของรูปสามเหลี่ยม $ABC$
	\end{problem}
	\begin{problem}
		หา $x^2+y^2+z^2+w^2$ ถ้า
		\begin{align*}
		\frac{x^2}{2^2-1^2} + \frac{y^2}{2^2-3^2} + \frac{z^2}{2^2-5^2} + \frac{w^2}{2^2-7^2} &= 1 \\
		\frac{x^2}{4^2-1^2} + \frac{y^2}{4^2-3^2} + \frac{z^2}{4^2-5^2} + \frac{w^2}{4^2-7^2} &= 1 \\
		\frac{x^2}{6^2-1^2} + \frac{y^2}{6^2-3^2} + \frac{z^2}{6^2-5^2} + \frac{w^2}{6^2-7^2} &= 1 \\
		\frac{x^2}{8^2-1^2} + \frac{y^2}{8^2-3^2} + \frac{z^2}{8^2-5^2} + \frac{w^2}{8^2-7^2} &= 1 		
		\end{align*}
	\end{problem}
	\begin{problem}
		หาค่า $(1-\frac{1}{1+2})(1-\frac{1}{1+2+3})...(1-\frac{1}{1+2+...+2563})$
	\end{problem}
	\begin{problem}
		จำนวนเต็มบวก $n$ ที่มีค่าน้อยที่สุด ซึ่งทำให้ $(2000n+1)(2008n+1)$ เป็นจำนวนกำลังสองสมบูรณ์
	\end{problem}
	\begin{problem}
		$x^3 + y^3 + (x + y)^3 + 30xy = 2000$ ค่าของ $x + y$
	\end{problem}
	\begin{problem}
		กำหนดให้ $f_0(x) = \frac{1}{1 - x}$ เเละ  $f_n(x) = f_0(f_{n-1}(x))$ สำหรับ $n \ge 1$ เเละ $x \ne 1$ จงหาค่าของ $f_{2002}(2002)$
	\end{problem}
	\begin{problem}
		$z_1,z_2,...,z_5$ เป็นจำนวนเชิงซ้อนที่เเตกต่างกัน เเละ $z_1+z_2+..+z_5 = 0$ เเละ $|z_1|=|z_2|=...=|z_5|$ จงหาส่วนจริงของ $\frac{z_1+z_2}{z_3}+\frac{z_2+z_3}{z_4}+\frac{z_3+z_4}{z_5}+\frac{z_4+z_5}{z_1}+\frac{z_5+z_1}{z_2}$
	\end{problem}
	\begin{problem}
		ถ้า $\frac{a-b}{b-c}+\frac{b-c}{c-a}+\frac{c-a}{a-b} = 5$ เเล้วจงหาค่าของ $(\frac{a-c}{b-c})^3+(\frac{b-a}{c-a})^3+(\frac{c-b}{a-b})^3$
	\end{problem}
	\begin{problem}
		ให้ $\frac{2^N+1}{641} = 409^2 +2556^2$ ถ้า $N$ เป็นจำนวนเต็มเเล้ว $N$ เท่ากับเท่าใด
	\end{problem}
	\begin{problem}
		จงหาผลรวมของจำนวนเต็มบวก $a$ ที่ทำให้ $\sqrt{(a+45)(a-5)}$ เป็นจำนวนเต็มบวก
	\end{problem}
	\begin{problem}
		จงหาคำตอบของสมการ $(1+\frac{1}{n})^{1+n} = (1+\frac{1}{9999})^{9999}$
	\end{problem}
	\begin{problem}
		กำหนดสี่เหลี่ยม $ABCD$ ซึ่ง $\angle{ABD} = \ang{38}, \angle{DBC} = \ang{46}, \angle{BCA} = \ang{22}$ เเละ $\angle{ACD} = \ang{48}$ จงเเสดงว่า $\angle{BDA} = \ang{18}$ 
	\end{problem}
	\begin{problem}
		สมมติ $x_1,x_2,..,x_49$ เป็นจำนวนจริงซึ่ง $x_1^2+2x_2^2+..+49x_{49}^2 = 1.$ หาค่าสูงสุดของ $x_1+2x_2+...49x_{49}$
	\end{problem}
	\begin{problem}
		ให้ $a,b,c$ เป็นจำนวนเต็มบวก จงเเสดงว่า ถ้า $\frac{a}{b}+\frac{b}{c}+\frac{c}{a}$ เป็นจำนวนเต็ม จงเเสดงว่า $abc$ เป็นกำลังสามสมบูรณ์ของจำนวนเต็ม
	\end{problem}
	\begin{problem}
		ให้ $x = (12112211122211112222)_3$ เเทนจำนวนในระบบฐาน $3$ ถ้า $x = (a_na_{n-1}a_{n-2}..a_1a_0)_9$ เเทนจำนวนในระบบฐาน $9$ เเละ $a_n \neq 0$ เเล้ว $a_n$ มีค่าเท่ากับเท่าใด
	\end{problem}
	\begin{problem}
		ให้ $ABC$ เป็นสามเหลี่ยมมุมฉากโดยที่ $AB=BC$ มี $P$ เป็นจุดภายใน $\bigtriangleup{ABC}$ โดยที่ $AP = 15cm,BP = 9cm, PC = 12 cm$ จงหาพื่นที่ $\bigtriangleup{ABC}$
	\end{problem}
	\begin{problem}
		หาผลเฉลยที่เป็นจำนวนเต็มของสมการ $(100-x)^2+(100-y)^2 = (x+y)^2$
	\end{problem}
	\begin{problem}
		ถ้ากำหนดสมการ $3^{2^{2x+1}-(9)2^{x+\frac{1}{2}}+32} = 27^{2^{x+\frac{1}{2}}}$ จงหาผลบวกของรากคำตอบของสมการ
	\end{problem}
	\begin{problem} ให้ $ABC$ เป็นสามเหลี่ยมมุมฉาก โดย $\angle{B} = \ang{90}$ มี $D$ เป็นจุดบนส่วนของเส้นตรง $AC$ ซึ่ง $AB = BC + CD$ ถ้า $\angle{BAC} = 2x$ เเละ $\angle{DBC} = 3x$ จงหา $x$ 
	\end{problem}
	\begin{problem}
		ให้ $\tau$ เป็นครึ่งวงกลมมีเส้นผ่านศูนย์กลาง $AB$ มีจุด $C$ อยู่บนส่วนของเส้นตรง $AB$ เเละ จุด $E$ เเละ $D$ อยู่บนคอร์ด $BA$ โดยที่ $E$ อยู่ระหว่าง $B$ เเละ $D$ ให้ เส้นสัมผัส $\tau$ ที่จุด $D$ เเละ $E$ ตัดกันที่จุด $F$ ถ้า $\angle{ACD}= \angle{ECB}$ จงเเสดงว่า $\angle{EFD} = \angle{ACD} + \angle{ECB}$
	\end{problem}
	\begin{problem}
		ถ้า $\frac{3xy}{x+y}=4, \frac{2yz}{y+z}=3, \frac{5xz}{x+z}=2$ เเล้วค่าของ $\frac{47xyz}{2xy+yz-4xz}$ เท่ากับเท่าใด
	\end{problem}
	\begin{problem} หาไตรอันดับของจำนวนนับ $(a,b,c)$ ซึ่ง $(2^a-1)(3^b-1)=c!$

	\end{problem}
	\begin{problem}
		จงหาผลคูณของจำนวนเต็ม $n$ ทั้งหมดที่ทำให้ $n^2+59n+881$ เป็นกำลังสองสมบูรณ์
	\end{problem}
	\begin{problem}
		จงหาจำนวนเต็มบวกที่น้อยที่สุดที่มีจำนวนตัวหารที่เป็นบวกเท่ากับ $24$ ตัว
	\end{problem}
	\begin{problem}
		ใน $ABC$ เป็นรูปสามเหลี่มเเนบในวงกลมที่มีจุด $O$ เป็นจุดศูนย์กลางถ้า $\angle{ABC} = \ang{70}, \angle{ACB} = \ang{50}$ เเละเส้นเเบ่งครึ่งมุม $BAC$ ตัดวงกลมที่จุด $D$ จงหาขนาดของ $\angle{ADO}$
	\end{problem}
	\begin{problem}
		พิจารณา $\bigtriangleup{ABC}$ จุด $M$ เป็นศูนย์กลางของ $AC$ วงกลมสัมผัสกับ $BC$ ที่จุด $B$ เเละผ่าน $M$ พบเส้นตรง $AB$ อีกครั้งที่จุด $P$ จงเเสดงว่า $(AB)(BP) = 2BM^2$
	\end{problem}
	\begin{problem}
		ให้ $ABCD$ เป็นสี่เหลี่ยมที่วงกลมล้อมรอบได้ ให้ $F$ เป็นจุดศูนย์กลางของคอร์ด $AB$ ของวงกลมของมันซึ่งไม่มี $C$ หรือ $D$ อยู่บน ให้ $DF$ เเละ $AC$ ตัดกันที่จุด $P$ เเละ $CF$ เเละ $BD$ ตัดกันที่จุด $Q$ จงเเสดงว่าเส้นตรง $PQ$ เเละ $AB$ ขนานกัน
	\end{problem}
	\begin{problem}
		จงหาจำนวนเฉพาะ $p$ ทั้งหมดที่ทำให้ $\frac{2^{p-1}-1}{p}$ เป็นกำลังสองสมบูรณ์
	\end{problem}
	\begin{problem}
		หาค่าของ $\sum_{k=1}^{11} \frac{\sin(\frac{2^{k+4}\pi}{89})}{\sin(\frac{2^k\pi}{89})}$
	\end{problem}
	\begin{problem}
		หาผลรวมของคำตอบที่เป็นจำนวนจึงซึ่งสอดคล้องกับสมการ  $$x^2+\cos{x} = 2563$$
	\end{problem}
	\begin{problem}
		ให้ $A$ เเละ $B$ เป็นจำนวนจริงซึ่ง $(x+1)^2$ หาร $x^{2017}+Ax+B$ ลงตัวเเล้ว$B$มีค่าเท่ากับเท่าไหร่
	\end{problem}
	\begin{problem}
		หาไตรอันดับ $(a,b,c)$ ที่เป็นจำนวนนับ ซึ่ง $$a^2+b^2=nlcm(a,b)+n^2$$ $$b^2+c^2=nlcm(b,c)+n^2$$ $$c^2+a^2=nlcm(c,a)+n^2$$ บาง $n$ ที่เป็นจำนวนนับ
	\end{problem}
	\begin{problem}
		หาจำนวนเฉพาะที่ใหญ่ที่สุด $p<300$ เเละ มีจำนวนเต็ม $x,y,u,v$ ซึ่ง $$p=x^2+y^2=u^2+7v^2$$
	\end{problem}
	\begin{problem}
		เลข $1,2,3,..,81$ ถูกสุ่มใส่ในตารางขนาด $9\times9$ จงเเสดงว่ามี ตารางย่อยขนาด $2\times2$ ซึ่งผลรวมของตัวเลขของมันมากกว่า $137$
	\end{problem}
	\begin{problem}
	ให้ $a, b$ เป็นจำนวนนับ จงเเสดงว่า $\frac{(2a)!(2b)!}{a!b!(a+b)!}$ เป็นจำนวนเต็ม
	\end{problem}
	\begin{problem}
		An experiment is performed in which each trial consists of tossing an ordinary six-sided die repeatedly and adding the numbers that come up; in each trial, as soon as the total exceeds $15,$ we stop tossing the die. For this experiment, what final total is expected to occur most often?
	\end{problem}
	\begin{problem}
		หาค่าต่ำสุดของ $c + \sqrt{3-c^3}$ ทุก ๆ $c>0$
	\end{problem}
	\begin{problem}
		ให้ $a, b$ เป็นจำนวนนับซึ่ง $a>b$ จงเเสดงว่า $\frac{a^2+b}{b^2-a^3}$ ไม่เป็นจำนวนเต็ม
	\end{problem}
	\begin{problem}
		$41$ rooks are placed on  a $10×10$ chesssbored Prove that you can choose five of them that do not attack each other
	\end{problem}
	\begin{problem}
		จงเเสดงว่าทุกจำนวนนับ $n,$ $\frac{1}{3^2}+\frac{1}{5^2}+...+\frac{1}{(2n+1)^2} < \frac{1}{4}$ 
	\end{problem}
	\begin{problem}
		ให้ $a,b,c,d$ เป็นจำนวนเต็มที่ไม่เป็น $0$ ที่เเตกต่างกัน ซึ่ง $a+b+c+d=0$ เเละ $M=(bc-ad)(ac-bd)(ab-cd)$ เเละ $96100<M<98000$ จงหาค่า $M$
	\end{problem}

\end{document}